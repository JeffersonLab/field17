\documentclass[12pt,epsf,here]{article}
\setlength{\textwidth}{6 in}
\setlength{\textheight}{9 in}
\setlength{\leftmargin}{0.7 in}
\setlength{\rightmargin}{0.7 in}
\newcommand{\h}{\hspace*{1 in}}
\input{epsf}
\begin{document}
\title{High Momentum Spectrometer(HMS) Magnet
\footnote{document source can be found at 
/home/cdaq/documents/slow\_controls/hmssource/hms.tex} }
\author{Sudhir Malik
\footnote{Slight modifications by David McKee to bring it up to date.
The bulk of this document was written in August 1998 by Sudhir Malik.}
}
\date{28 Febuary 1999}
\maketitle
%\begin{abstract}
%\end {abstract}
\section{Overview}

The HMS is composed of three superconducting quadrupole magnets, Q1, Q2,
and Q3, and one superconducting dipole magnet The quadrupole magnets are
referred to as Q1, Q2, and Q3, where a particle first traverses Q1, then Q2
and Q3, and finally traverses the dipole magnet.  All of them have
independent power supplies controlled by separate PC computers.\textbf{
Access to these computers is through a display screen(HMS control Screen)
in the counting house for our purpose.}


\section{How to determine settings for magnet(field program)}

To know the settings for the magnets(Q1,Q2,Q3 \& Dipole) one has to look
at the screen corresponding to each of them. To select a page i.e. a
magnet do the following \\
\\
\h{Press and hold "Num lock"}\\
\h{Press and hold "minus(-)"}\\
\h{Release        "minus(-)}\\
\h{Release        "Num lock"}\\
\\
and then press any one of the following key\\
\\
\h{"A" for Q1}\\
\h{"B" for Q2}\\
\h{"C" for Q3}\\
\h{"D" for Q4}\\
\\
and then\\
\\
\h{Press "Return"}\\
\\
If you want the screens for the the quads and dipole to be display one
after another in continuous manner i.e CYCLE , press
\\
\h{"KG"} to start CYCLE
\\
To stop it
\\
\h{"KH"} to stop CYCLE

\textbf{Do not confuse this word ``CYCLE'' with ``CYCLING THE MAGNETS'' 
which is explained in section 3}
\\
\section{How to change the HMS Magnet Central Momentum value}

To obtain the predicted current or field settings do the following:
\\
In \textbf{cdaqh1(the screen is probably already opened) } from the home
directory go to the directory /cdaq/FIELD.\\
\\
then type ``field99'' against the prompt.\\
\\
This will display the following message\\
\begin{verbatim}Input HMS momentum( i.e. central momentum) in GeV/c,electron 
is negative: \end{verbatim}  

Type the momentum of the particle: \textbf{for electron always include the 
negative sign in front of the value i.e. say for e.g. type the value
-2.455 and not 2.455}
\\
This will then give for the following result for -2.455 GeV/c.\\
\begin{verbatim}
 Set current for Q1 =  318.8337885684941 amps
 Set current for Q2 =  -253.45770666407 amps
 Set current for Q3 =  124.5714821214358 amps
 Set NMR for Dipole =  -.672752725470763 Tesla
\end{verbatim}
\textbf{Note that the current values are in Amperes and the NMR field is in
Tesla.}\\
\\
Thus you will obtain the above set of values that correspond to  a
required setting of HMS central momentum. Now type in these values against
the command line in the display screen for each of the magnet as
given in section 4.Note that for Q1,Q2 and Q3 you get current values while
for Dipole you get a NMR field value. 

\section{How to cycle the HMS Magnets with above obtained values }

First select a magnet as described above in Section 2. This particular 
screen will have, say for Q2, the following heading "Q2 PSU Display
Screen". PSU stands for Power Supply Unit.Then for each of them do the
following:

\subsection{For Quadrupoles(Q1,Q2,Q3)}

The quadrupoles suffer from some hysterisis effects. Jochen Volmer designed 
and tested a procedure for setting the quadrupoles that achieved a very 
high degree of reproducablility in setting the quads at low current. 
The procedure given here is taken from that procedure, but has not yet 
been tested.\\
\textbf{The procedure (tm):}
\begin{enumerate}
	\item{On every change of polarity, take the magnet up to 950 Amps 
	(in the new polarity!), then down to zero before setting the
	current.}
	\item{To set the current the first time after a polarity change 
	(or when you're not sure if it's been done already), 
	go up to 950 Amps, then down to the desired current.
	
	This means: to change the polarity and set the current go to 950 Amps,
	down to zero, back up to 950 Amps, and down to the desired setting.}
	\item{Subsequently:
	\begin{itemize}
		\item{Changes to lower currents can be made directly. 
		I.e. just set the magnet for the lower current.}
		\item{For changes to higher current, first overshoot 
		by 100 Amps, then come back down to the desired current.}
	\end{itemize}}
\end{enumerate}
This procedure is called \textbf{CYCLING THE MAGNET}, and needs to be 
followed for all three quadrupoles.

%For Q1,Q2 and Q3 do the following. At this stage you already have the current
%values calculated for each of them from section 3. 
%For example you want to feed-in 318.91 Amperes for Q1. 
%In order to do this first feed-in a value that is
%about 100 Amperes higher than this value i.e. feed-in 418.91 Amperes and after
%this value is achieved then feed-in the set value i.e. 318.91 Amperes. 

 
%Do the same procedure 
%for Q2 and Q3. \\
%\\

A current can be selected by typing-in 
\begin{verbatim}psu:setcur:\#.\#\#. \end{verbatim} 
against the command line. While typing in the current values, 
the negative sign for the current value is taken care of automatically. 
So basically you type in the number only. 
After typing in the current value press $<enter>$, 
then go to the next screen/magnet as explained in section 2.\\
\\
There is a readback of the set current from the power supply.The MPS
voltage and current are also displayed to further indicate where ramp   
up/down is in progress.There is a list of MPS interlocks displayed to help
understand trips.\\

\subsection{For Dipole}
Use the Field Setting Procedure for the Dipole. This will bring the
magnet to a field near but not equal to the desired field.   
Turn field control on for the Dipole. The magnet will go to the
desired field.Wait at least 7 minutes for the dipole magnet to settle.\\
\\
\textbf{1.Be patient to have the response of what you type in. Do not go
on typing command again and again. The response takes a while.}\\
\\
\textbf{2.Sometimes you might need to type in twice a particular current
value. Again the keyword is PATIENCE}\\



\section{What routine checks to make}
\subsection{checking cryogenics}
Click on the He(for Liquid helium) and N2(for liquid nitrogen) from the   
menu in the upper right corner of the screen for their status display for
a given magnet\\
\\
The HMS magnets all operate with liquid level control of a reservoir. It
is therefore sufficient to verify that the liquid level is near the set  
point to be assured of cryogenic happiness.  The liquid level is normally
within a few \% of these values. If the level is significantly above [85%
for quads or 94\% for the dipole] the helium reservoirs are overfilling.  
This is not harmful and the levels will return to normal in several hours.
If the level is significantly below the set points (5% or more) there is
usually something wrong.  Selecting a time graph of liquid level is
helpful in determining if the situation is a temporary fluctuation or if
the situation is serious.

\subsection{Helium Problem Resolution}

If helium liquid level is observed falling, check all four systems. If all
four systems are losing then call CHL x7405 as the likely cause is a site
wide problem. CHL will advise if recovery is short (1-2 hours) or much   
longer. If the recovery is short do nothing! If the recovery is long then 
it can be beneficial to make some adjustments in Hall C. This requires an 
access and a knowledgeable individual, the on-call cryo operator should be
summoned.

\subsection{Single System Failures}

\subsubsection{Single System Loss of L$N_2$}

If a single system is observed losing LN you can wait until the next day
to call someone in as the LN usage of all the magnets is extremely low.
They can go for 24 hours without a refill.

\subsubsection{Single System Loss of LHe Level}

This is usually caused by a single computer failure or components failure.
Call the cryo operator and plan an access to Hall C. The dipole reservoir
will go empty in 1 hour so a quick reaction is necessary. The quads take
much longer, 4 hours or more to empty allowing more time to react. All of
the magnets have low level interlocks so that you can safely operate until
they are ``dry."

\subsection{Temporary Loss of L$N_2$ To All Systems}  

Occasionally during site LN delivery, the supply to Hall C is temporarily
stopped. This can be checked by calling CHL x7390. There can be local Hall
C problems that result in loss of LN to the magnets. The ``call in" can be
deferred to a convenient time for this kind of problem.
\subsection{Typical LHe and L$N_2$ \% at a glance}

\begin{tabular}{|c|c|c|} \hline
Magnet&Liquid $N_{2}$&Liquid He \\ \hline
Q1&75\%&75\% \\ \hline
Q2&75\%&75\% \\ \hline
Q3&75\%&75\% \\ \hline
Dipole&70\%&70\%/65\%(Hi/Low)\\ \hline
\end{tabular}

\section{Who to call with problems}
\begin{tabbing}
aaaaaaaaaaaaaaaaaaaa \=aaaaaaaa \=aaaaaaa \=aaaaaa    \kill
\textbf{Name}    \> \textbf{Phone}   \>\textbf{Page}   \> \textbf{Home}\\
\\
\textbf{Paul Brindza} \>7588 \>7588 \>898-3734\\
\textbf{Mike Fowler} \>7162 \>7162 \>875-0074\\
\textbf{Steve Lassiter} \>7129 \>7129 \>919-465-4348\\
\end{tabbing}


\subsection{HMS-Magnets and Power Supplies-general information}

\subsubsection{Magnets}
The HMS magnets are all superconducting and hence their coils must be
maintained at cryogenic temperatures during operations. All the HMS
magnets cryogenic services are supplied through the overhead cryogenic
lines. 

Cryogenic information about each magnet is available on the control
screens downstairs, one for each magnet. These are located on the first
deck of the shield house by the ``pasta fork" at the rear of the
spectrometer. \textbf{During run periods the control screens are sent
upstairs to the Hall C counting house and information on all the HMS
magnets is available on the HMS control screen located in the center of
the main console.}
\subsubsection{Power supplies}

The power supplies for the magnets are located on the carriage adjacent to
the magnets. The supplies are all water cooled and the water flow rate to
the supplies can be seen on the water flow meter located near the
electronics boxes on the floor near the pivot. Under normal conditions the
meter should read approximately 33 . This meter views the flow for all the
HMS power supplies (dipole and quads) and a reading of 33 corresponds to
approximately 20 gallons per minute through the combination of supplies
(they are supplied in parallel).

The front panel of the power supplies are interlocked. Under no
circumstances should the front panel of any supply be opened by anyone
other than authorized personnel. \textbf{There is a keyed electrical
interlock located in the Hall C counting house main console to prevent the
power supplies from being energized at inappropriate times.} Personnel
listed in the Responsible Personnel manuscript maintain a copy of the key. 

When the supplies are energized there are flashing red lights placed at
several locations on the HMS carriage to alert personnel to the magnet
status. There are also signs posted listing the dangers of high magnetic
fields. 

\textbf{The control interface for the power supplies is available through
the HMS control screen in the Hall C counting house.} 

\subsection{ HMS-Quadrupole Magnets}
The quadrupoles determine the transverse focusing properties of the
spectrometer and to a large extent its acceptance. There are currently no
operating limits imposed on the quads.\\
\\
All three quadrupoles for the HMS spectrometer are cold iron
superconducting magnets. The soft iron around the superconducting coil
enhances the field at the coil center and reduces stray fields. The basic
parameters for the first quadrupole, Q1, are an effective (actual) length
of 1.89 (2.34) meter and an inner pole radius of 25.0 centimeter. [10] The
vacuum vessel inner radius for Q1 is 20.05 cm. To achieve the lowest
possible angle setting of the HMS spectrometer (with respect to the beam
line), Q1 is made asymmetrical, and is elongated in the vertical
direction. For the same reason a notch in the outer mantle of the Q1 cryo
vessel is made, such that the incident electron beam passes through this
notch when the HMS spectrometer is at its smallest angle of 12.5 degrees.
The other two quadrupoles, Q2 and Q3, are essentially identical with an
effective (actual) length of about 2.10 (2.60) meter and an inner pole
radius of 35.0 centimeter. For these quadrupoles the vacuum vessel inner
radius amounts to 30.0 centimeter.\\
\\
The maximum operating currents (assuming a 4 GeV/c momentum particle) for
the quadrupoles are about 580 A, 440 A, and 220 A, for Q1, Q2, and Q3,
respectively. To establish a correct focusing onto the detector plane with
the quadrupole triplet we may want to cycle the quadrupoles to about 20%
higher current values, rendering maximum currents of 700 A, 530 A, and 270
A, respectively. This will render pole field values of 1.25, 1.30, and
0.65 T, respectively. The energy stored in the quadrupole fields is
sufficient to cause an unrecoverable quench if all the energy stored is
dumped into the magnets. [11] Therefore a quench protection circuit is
incorporated. However, a quench can only happen if the cryomagnets have a
helium level below the coil during operation.\\
\\
The operating current to the quadrupole coils is provided by three
Danfysik System 8000 power supplies, which can operate up to 1250 A
current and 5 V voltage. The power supplies will be cooled with a combined
maximum water flow of 45 liters per minute.\\ 
\\
In addition to the main quadrupole windings, all quadrupoles have
multipole windings.  To further optimize focusing properties of the HMS
magnet system we intend to operate including some of these multipole trim
coils. [12] The operating current for these multipole corrections is
small, only (the multipole corrections are typically less than 2% of the
main quadrupole field), of order 50 A, and will be provided by three HP
power supplies. These power supplies can operate up to 100 A current and 5
V voltage.\\
\\
\subsection{HMS - Dipole Magnet}
The dipole is the dispersive element in the system determines the
central momentum of the spectrometer. The present operations envelope
states that the supply may not be operated at a currents above 1300 Amps.
This corresponds to a central momentum of 4 GeV/c.\\
\\
The dipole for the HMS spectrometer is a superconducting, cryostable
magnet. Its basic parameters are an effective length of 5.26 meter, a bend
radius of 12.06 meter, and a gap width of 42 cm. Its actual size is 5.99
meter long, 2.75 meter wide, and 4.46 meter high. It is configured to
achieve a 25 degree bending angle for 4 GeV/c momentum particles at a
central field excitation of 1.11 T. For the HMS dipole to reach 1.11 T the
maximum operating current for the coil amounts to 1300 A.\\ 
\\
The dipole has been designed to achieve cryostability up to a field of 2
T, and this property has been extensively tested up to a field of 1.11 T.
The cryostable coils are equipped with an energy removal circuit to cover
the possibility of an unrecoverable quench. However, this can only happen
if the helium level drops below the coil during operation. The current to
the coils will be provided by a Danfysik System 8000 power supply, which
can operate up to 3000 A current and 10 V voltage. This power supply is
located on the carriage beside the dipole, and will be cooled with a
maximum water flow of 35 liters per minute. The flow of the magnet cooling
water will be regulated by flow meters installed on the floor of Hall C.
The total water flow needed to cool the 4 power supplies for the HMS
magnet system (dipole and quadrupoles) amounts to 80 liters per minute,
with a supply pressure of cooling water for Hall C of 250 psi.\\
\\

\section{Sample screens}

In the following pages are shown typical copies of the various HMS
superconducting magnet control screens.  These screens can be checked when
scrolling through the pages of the magnet control program.  These typical
examples may be used as references for comparing current values to
determine if operations are reasonable. 

Small deviations are not to be considered as cause for alarm but rather
should prompt the operator to look at the trend graph for the variable in
question. This is at present the best way to determine if the cryogenic
state is deteriorating or is merely just slightly different. The Key
Operators for the HMS magnets will make adjustments to the control
parameters to suit changing situations but these are usually small (at the
few percent level) changes in valve limits or setpoints.\\
\\
\textbf{For detailed information see the following web site}\\
\begin{verbatim}http://www.jlab.org/Hall-C/document/manual3/manual3.html \end{verbatim}


\begin{figure}[htbp]
\centerline{\epsfxsize=15.cm \epsfysize=15cm \epsfbox{hms1.ps} }
\textbf{Figure: Sample Helium Monitoring Page - Q3} 
\end{figure}

\begin{figure}[htbp]
\centerline{\epsfxsize=13.cm \epsfysize=10cm \epsfbox{hms2.ps} }
\textbf{Figure: Sample Nitrogen Monitoring Page - Q2 }
\end{figure}

\begin{figure}[htbp]
\centerline{\epsfxsize=13.cm \epsfysize=10cm \epsfbox{hms3.ps} }
\textbf{Figure: Sample Power Supply Monitoring Page - Q2 }
\end{figure}

\begin{figure}[htbp]
\centerline{\epsfxsize=13.cm \epsfysize=10cm \epsfbox{hms4.ps} }
\textbf{Figure: Sample Interlock Monitoring Page - Q3  }
\end{figure}



\begin{figure}[htbp]
\centerline{\epsfxsize=13.cm \epsfysize=10cm \epsfbox{hms5.ps} }
\textbf{ Figure: Q1 Valve Control Page }  
\end{figure}

\begin{figure}[htbp]
\centerline{\epsfxsize=13.cm \epsfysize=10cm \epsfbox{hms6.ps} }
\textbf{Figure: Q2 Valve Control Page }
\end{figure}

\begin{figure}[htbp]
\centerline{\epsfxsize=13.cm \epsfysize=10cm \epsfbox{hms7.ps} }
\textbf{ Figure: Q3 Valve Control Page }
\end{figure}

\begin{figure}[htbp]
\centerline{\epsfxsize=13.cm \epsfysize=10cm \epsfbox{hms8.ps} }
\textbf{Figure: Sample Magnet Temperature Display - Q3 }
\end{figure}


\begin{figure}[htbp]
\centerline{\epsfxsize=13.cm \epsfysize=10cm \epsfbox{hms9.ps} }
\textbf{Figure: Sample Coil Voltage Monitoring Page - Q2 }
\end{figure}

\begin{figure}[htbp]
\centerline{\epsfxsize=13.cm \epsfysize=10cm \epsfbox{hms10.ps} }
\textbf{Figure: Sample Trend Page }
\end{figure}

\end{document}









